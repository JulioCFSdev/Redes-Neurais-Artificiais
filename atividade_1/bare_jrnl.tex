%% Support sites:
%% http://www.michaelshell.org/tex/ieeetran/
%% http://www.ctan.org/pkg/ieeetran
%% and
%% http://www.ieee.org/

\documentclass[journal]{IEEEtran}
\usepackage[brazil]{babel}
\usepackage[utf8]{inputenc}
\usepackage{url}
\usepackage{xcolor}
\usepackage{cite}
\usepackage{booktabs}
\usepackage{icomma}
\usepackage{amsmath,amsthm,amssymb}
\usepackage{graphicx}




\begin{document}

\title{Título do Trabalho}


\author{Aluno 1, Aluno 2\\
\thanks{E-mails: \texttt{aluno1.eng@uea.edu.br, aluno2.snf@uea.edu.br}}}

\markboth{Redes Neurais Artificiais 2022.2 -- Elloá B. Guedes}%
{Shell \MakeLowercase{\textit{et al.}}: Bare Demo of IEEEtran.cls for IEEE Journals}
\maketitle
\IEEEpeerreviewmaketitle

\section{Descrição da Atividade}

Os alunos da turma devem organizar-se em duplas e utilizar o \emph{template} \LaTeX do IEEE para produzir um texto dissertativo argumentativo (máximo de $2$ laudas já incluindo referências bibliográficas), que responda aos seguintes questionamentos a partir das informações extraídas da leitura do artigo intitulado ``\emph{How to Read Articles That Use Machine Learning}'' \cite{Liu2019}.

\begin{enumerate}
    \item Por que é importante que profissionais da área de Saúde conheçam termos técnicos da área de \emph{Machine Learning} e saibam avaliar soluções dessa natureza?
    \item Qual a diferença entre Aprendizado Supervisionado e Aprendizado Não-Supervisionado? Cite um exemplo de tarefa segundo cada paradigma.
    \item Quais as diferenças entre os métodos de \emph{Machine Learning} tradicionais e o mais recentemente propostos na literatura? Quais as vantagens e limitações de cada um deles?
    \item O que é regularização?  Como ela influencia no desenvolvimento de modelos de \emph{Machine Learning}? Cite um exemplo de técnica dessa natureza.
    \item Quais estratégias podem ser utilizadas para detectar e prevenir \emph{overfitting}?
    \item Como assegurar o desenvolvimento de bons modelos de \emph{Machine Learning} para problemas reais, inclusive na área de Saúde?
\end{enumerate}

A atividade será avaliada no tocante aos seguintes critérios: uso da norma culta da língua; respeito aos padrões do \emph{template} do \LaTeX; corretude, completude, qualidade e diversidade nas respostas; fluidez no texto, uso de conectivos, coesão, coerência e objetividade; e, uso de referências bibliográficas auxiliares relevantes, caso necessário.







\bibliographystyle{IEEEtran}
\bibliography{ref}

\end{document}


